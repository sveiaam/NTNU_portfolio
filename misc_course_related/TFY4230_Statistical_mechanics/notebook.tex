
% Default to the notebook output style

    


% Inherit from the specified cell style.




    
\documentclass[11pt]{article}

    
    
    \usepackage[T1]{fontenc}
    % Nicer default font (+ math font) than Computer Modern for most use cases
    \usepackage{mathpazo}

    % Basic figure setup, for now with no caption control since it's done
    % automatically by Pandoc (which extracts ![](path) syntax from Markdown).
    \usepackage{graphicx}
    % We will generate all images so they have a width \maxwidth. This means
    % that they will get their normal width if they fit onto the page, but
    % are scaled down if they would overflow the margins.
    \makeatletter
    \def\maxwidth{\ifdim\Gin@nat@width>\linewidth\linewidth
    \else\Gin@nat@width\fi}
    \makeatother
    \let\Oldincludegraphics\includegraphics
    % Set max figure width to be 80% of text width, for now hardcoded.
    \renewcommand{\includegraphics}[1]{\Oldincludegraphics[width=.8\maxwidth]{#1}}
    % Ensure that by default, figures have no caption (until we provide a
    % proper Figure object with a Caption API and a way to capture that
    % in the conversion process - todo).
    \usepackage{caption}
    \DeclareCaptionLabelFormat{nolabel}{}
    \captionsetup{labelformat=nolabel}

    \usepackage{adjustbox} % Used to constrain images to a maximum size 
    \usepackage{xcolor} % Allow colors to be defined
    \usepackage{enumerate} % Needed for markdown enumerations to work
    \usepackage{geometry} % Used to adjust the document margins
    \usepackage{amsmath} % Equations
    \usepackage{amssymb} % Equations
    \usepackage{textcomp} % defines textquotesingle
    % Hack from http://tex.stackexchange.com/a/47451/13684:
    \AtBeginDocument{%
        \def\PYZsq{\textquotesingle}% Upright quotes in Pygmentized code
    }
    \usepackage{upquote} % Upright quotes for verbatim code
    \usepackage{eurosym} % defines \euro
    \usepackage[mathletters]{ucs} % Extended unicode (utf-8) support
    \usepackage[utf8x]{inputenc} % Allow utf-8 characters in the tex document
    \usepackage{fancyvrb} % verbatim replacement that allows latex
    \usepackage{grffile} % extends the file name processing of package graphics 
                         % to support a larger range 
    % The hyperref package gives us a pdf with properly built
    % internal navigation ('pdf bookmarks' for the table of contents,
    % internal cross-reference links, web links for URLs, etc.)
    \usepackage{hyperref}
    \usepackage{longtable} % longtable support required by pandoc >1.10
    \usepackage{booktabs}  % table support for pandoc > 1.12.2
    \usepackage[inline]{enumitem} % IRkernel/repr support (it uses the enumerate* environment)
    \usepackage[normalem]{ulem} % ulem is needed to support strikethroughs (\sout)
                                % normalem makes italics be italics, not underlines
    

    
    
    % Colors for the hyperref package
    \definecolor{urlcolor}{rgb}{0,.145,.698}
    \definecolor{linkcolor}{rgb}{.71,0.21,0.01}
    \definecolor{citecolor}{rgb}{.12,.54,.11}

    % ANSI colors
    \definecolor{ansi-black}{HTML}{3E424D}
    \definecolor{ansi-black-intense}{HTML}{282C36}
    \definecolor{ansi-red}{HTML}{E75C58}
    \definecolor{ansi-red-intense}{HTML}{B22B31}
    \definecolor{ansi-green}{HTML}{00A250}
    \definecolor{ansi-green-intense}{HTML}{007427}
    \definecolor{ansi-yellow}{HTML}{DDB62B}
    \definecolor{ansi-yellow-intense}{HTML}{B27D12}
    \definecolor{ansi-blue}{HTML}{208FFB}
    \definecolor{ansi-blue-intense}{HTML}{0065CA}
    \definecolor{ansi-magenta}{HTML}{D160C4}
    \definecolor{ansi-magenta-intense}{HTML}{A03196}
    \definecolor{ansi-cyan}{HTML}{60C6C8}
    \definecolor{ansi-cyan-intense}{HTML}{258F8F}
    \definecolor{ansi-white}{HTML}{C5C1B4}
    \definecolor{ansi-white-intense}{HTML}{A1A6B2}

    % commands and environments needed by pandoc snippets
    % extracted from the output of `pandoc -s`
    \providecommand{\tightlist}{%
      \setlength{\itemsep}{0pt}\setlength{\parskip}{0pt}}
    \DefineVerbatimEnvironment{Highlighting}{Verbatim}{commandchars=\\\{\}}
    % Add ',fontsize=\small' for more characters per line
    \newenvironment{Shaded}{}{}
    \newcommand{\KeywordTok}[1]{\textcolor[rgb]{0.00,0.44,0.13}{\textbf{{#1}}}}
    \newcommand{\DataTypeTok}[1]{\textcolor[rgb]{0.56,0.13,0.00}{{#1}}}
    \newcommand{\DecValTok}[1]{\textcolor[rgb]{0.25,0.63,0.44}{{#1}}}
    \newcommand{\BaseNTok}[1]{\textcolor[rgb]{0.25,0.63,0.44}{{#1}}}
    \newcommand{\FloatTok}[1]{\textcolor[rgb]{0.25,0.63,0.44}{{#1}}}
    \newcommand{\CharTok}[1]{\textcolor[rgb]{0.25,0.44,0.63}{{#1}}}
    \newcommand{\StringTok}[1]{\textcolor[rgb]{0.25,0.44,0.63}{{#1}}}
    \newcommand{\CommentTok}[1]{\textcolor[rgb]{0.38,0.63,0.69}{\textit{{#1}}}}
    \newcommand{\OtherTok}[1]{\textcolor[rgb]{0.00,0.44,0.13}{{#1}}}
    \newcommand{\AlertTok}[1]{\textcolor[rgb]{1.00,0.00,0.00}{\textbf{{#1}}}}
    \newcommand{\FunctionTok}[1]{\textcolor[rgb]{0.02,0.16,0.49}{{#1}}}
    \newcommand{\RegionMarkerTok}[1]{{#1}}
    \newcommand{\ErrorTok}[1]{\textcolor[rgb]{1.00,0.00,0.00}{\textbf{{#1}}}}
    \newcommand{\NormalTok}[1]{{#1}}
    
    % Additional commands for more recent versions of Pandoc
    \newcommand{\ConstantTok}[1]{\textcolor[rgb]{0.53,0.00,0.00}{{#1}}}
    \newcommand{\SpecialCharTok}[1]{\textcolor[rgb]{0.25,0.44,0.63}{{#1}}}
    \newcommand{\VerbatimStringTok}[1]{\textcolor[rgb]{0.25,0.44,0.63}{{#1}}}
    \newcommand{\SpecialStringTok}[1]{\textcolor[rgb]{0.73,0.40,0.53}{{#1}}}
    \newcommand{\ImportTok}[1]{{#1}}
    \newcommand{\DocumentationTok}[1]{\textcolor[rgb]{0.73,0.13,0.13}{\textit{{#1}}}}
    \newcommand{\AnnotationTok}[1]{\textcolor[rgb]{0.38,0.63,0.69}{\textbf{\textit{{#1}}}}}
    \newcommand{\CommentVarTok}[1]{\textcolor[rgb]{0.38,0.63,0.69}{\textbf{\textit{{#1}}}}}
    \newcommand{\VariableTok}[1]{\textcolor[rgb]{0.10,0.09,0.49}{{#1}}}
    \newcommand{\ControlFlowTok}[1]{\textcolor[rgb]{0.00,0.44,0.13}{\textbf{{#1}}}}
    \newcommand{\OperatorTok}[1]{\textcolor[rgb]{0.40,0.40,0.40}{{#1}}}
    \newcommand{\BuiltInTok}[1]{{#1}}
    \newcommand{\ExtensionTok}[1]{{#1}}
    \newcommand{\PreprocessorTok}[1]{\textcolor[rgb]{0.74,0.48,0.00}{{#1}}}
    \newcommand{\AttributeTok}[1]{\textcolor[rgb]{0.49,0.56,0.16}{{#1}}}
    \newcommand{\InformationTok}[1]{\textcolor[rgb]{0.38,0.63,0.69}{\textbf{\textit{{#1}}}}}
    \newcommand{\WarningTok}[1]{\textcolor[rgb]{0.38,0.63,0.69}{\textbf{\textit{{#1}}}}}
    
    
    % Define a nice break command that doesn't care if a line doesn't already
    % exist.
    \def\br{\hspace*{\fill} \\* }
    % Math Jax compatability definitions
    \def\gt{>}
    \def\lt{<}
    % Document parameters
    \title{n chain Ising model}
    
    
    

    % Pygments definitions
    
\makeatletter
\def\PY@reset{\let\PY@it=\relax \let\PY@bf=\relax%
    \let\PY@ul=\relax \let\PY@tc=\relax%
    \let\PY@bc=\relax \let\PY@ff=\relax}
\def\PY@tok#1{\csname PY@tok@#1\endcsname}
\def\PY@toks#1+{\ifx\relax#1\empty\else%
    \PY@tok{#1}\expandafter\PY@toks\fi}
\def\PY@do#1{\PY@bc{\PY@tc{\PY@ul{%
    \PY@it{\PY@bf{\PY@ff{#1}}}}}}}
\def\PY#1#2{\PY@reset\PY@toks#1+\relax+\PY@do{#2}}

\expandafter\def\csname PY@tok@w\endcsname{\def\PY@tc##1{\textcolor[rgb]{0.73,0.73,0.73}{##1}}}
\expandafter\def\csname PY@tok@c\endcsname{\let\PY@it=\textit\def\PY@tc##1{\textcolor[rgb]{0.25,0.50,0.50}{##1}}}
\expandafter\def\csname PY@tok@cp\endcsname{\def\PY@tc##1{\textcolor[rgb]{0.74,0.48,0.00}{##1}}}
\expandafter\def\csname PY@tok@k\endcsname{\let\PY@bf=\textbf\def\PY@tc##1{\textcolor[rgb]{0.00,0.50,0.00}{##1}}}
\expandafter\def\csname PY@tok@kp\endcsname{\def\PY@tc##1{\textcolor[rgb]{0.00,0.50,0.00}{##1}}}
\expandafter\def\csname PY@tok@kt\endcsname{\def\PY@tc##1{\textcolor[rgb]{0.69,0.00,0.25}{##1}}}
\expandafter\def\csname PY@tok@o\endcsname{\def\PY@tc##1{\textcolor[rgb]{0.40,0.40,0.40}{##1}}}
\expandafter\def\csname PY@tok@ow\endcsname{\let\PY@bf=\textbf\def\PY@tc##1{\textcolor[rgb]{0.67,0.13,1.00}{##1}}}
\expandafter\def\csname PY@tok@nb\endcsname{\def\PY@tc##1{\textcolor[rgb]{0.00,0.50,0.00}{##1}}}
\expandafter\def\csname PY@tok@nf\endcsname{\def\PY@tc##1{\textcolor[rgb]{0.00,0.00,1.00}{##1}}}
\expandafter\def\csname PY@tok@nc\endcsname{\let\PY@bf=\textbf\def\PY@tc##1{\textcolor[rgb]{0.00,0.00,1.00}{##1}}}
\expandafter\def\csname PY@tok@nn\endcsname{\let\PY@bf=\textbf\def\PY@tc##1{\textcolor[rgb]{0.00,0.00,1.00}{##1}}}
\expandafter\def\csname PY@tok@ne\endcsname{\let\PY@bf=\textbf\def\PY@tc##1{\textcolor[rgb]{0.82,0.25,0.23}{##1}}}
\expandafter\def\csname PY@tok@nv\endcsname{\def\PY@tc##1{\textcolor[rgb]{0.10,0.09,0.49}{##1}}}
\expandafter\def\csname PY@tok@no\endcsname{\def\PY@tc##1{\textcolor[rgb]{0.53,0.00,0.00}{##1}}}
\expandafter\def\csname PY@tok@nl\endcsname{\def\PY@tc##1{\textcolor[rgb]{0.63,0.63,0.00}{##1}}}
\expandafter\def\csname PY@tok@ni\endcsname{\let\PY@bf=\textbf\def\PY@tc##1{\textcolor[rgb]{0.60,0.60,0.60}{##1}}}
\expandafter\def\csname PY@tok@na\endcsname{\def\PY@tc##1{\textcolor[rgb]{0.49,0.56,0.16}{##1}}}
\expandafter\def\csname PY@tok@nt\endcsname{\let\PY@bf=\textbf\def\PY@tc##1{\textcolor[rgb]{0.00,0.50,0.00}{##1}}}
\expandafter\def\csname PY@tok@nd\endcsname{\def\PY@tc##1{\textcolor[rgb]{0.67,0.13,1.00}{##1}}}
\expandafter\def\csname PY@tok@s\endcsname{\def\PY@tc##1{\textcolor[rgb]{0.73,0.13,0.13}{##1}}}
\expandafter\def\csname PY@tok@sd\endcsname{\let\PY@it=\textit\def\PY@tc##1{\textcolor[rgb]{0.73,0.13,0.13}{##1}}}
\expandafter\def\csname PY@tok@si\endcsname{\let\PY@bf=\textbf\def\PY@tc##1{\textcolor[rgb]{0.73,0.40,0.53}{##1}}}
\expandafter\def\csname PY@tok@se\endcsname{\let\PY@bf=\textbf\def\PY@tc##1{\textcolor[rgb]{0.73,0.40,0.13}{##1}}}
\expandafter\def\csname PY@tok@sr\endcsname{\def\PY@tc##1{\textcolor[rgb]{0.73,0.40,0.53}{##1}}}
\expandafter\def\csname PY@tok@ss\endcsname{\def\PY@tc##1{\textcolor[rgb]{0.10,0.09,0.49}{##1}}}
\expandafter\def\csname PY@tok@sx\endcsname{\def\PY@tc##1{\textcolor[rgb]{0.00,0.50,0.00}{##1}}}
\expandafter\def\csname PY@tok@m\endcsname{\def\PY@tc##1{\textcolor[rgb]{0.40,0.40,0.40}{##1}}}
\expandafter\def\csname PY@tok@gh\endcsname{\let\PY@bf=\textbf\def\PY@tc##1{\textcolor[rgb]{0.00,0.00,0.50}{##1}}}
\expandafter\def\csname PY@tok@gu\endcsname{\let\PY@bf=\textbf\def\PY@tc##1{\textcolor[rgb]{0.50,0.00,0.50}{##1}}}
\expandafter\def\csname PY@tok@gd\endcsname{\def\PY@tc##1{\textcolor[rgb]{0.63,0.00,0.00}{##1}}}
\expandafter\def\csname PY@tok@gi\endcsname{\def\PY@tc##1{\textcolor[rgb]{0.00,0.63,0.00}{##1}}}
\expandafter\def\csname PY@tok@gr\endcsname{\def\PY@tc##1{\textcolor[rgb]{1.00,0.00,0.00}{##1}}}
\expandafter\def\csname PY@tok@ge\endcsname{\let\PY@it=\textit}
\expandafter\def\csname PY@tok@gs\endcsname{\let\PY@bf=\textbf}
\expandafter\def\csname PY@tok@gp\endcsname{\let\PY@bf=\textbf\def\PY@tc##1{\textcolor[rgb]{0.00,0.00,0.50}{##1}}}
\expandafter\def\csname PY@tok@go\endcsname{\def\PY@tc##1{\textcolor[rgb]{0.53,0.53,0.53}{##1}}}
\expandafter\def\csname PY@tok@gt\endcsname{\def\PY@tc##1{\textcolor[rgb]{0.00,0.27,0.87}{##1}}}
\expandafter\def\csname PY@tok@err\endcsname{\def\PY@bc##1{\setlength{\fboxsep}{0pt}\fcolorbox[rgb]{1.00,0.00,0.00}{1,1,1}{\strut ##1}}}
\expandafter\def\csname PY@tok@kc\endcsname{\let\PY@bf=\textbf\def\PY@tc##1{\textcolor[rgb]{0.00,0.50,0.00}{##1}}}
\expandafter\def\csname PY@tok@kd\endcsname{\let\PY@bf=\textbf\def\PY@tc##1{\textcolor[rgb]{0.00,0.50,0.00}{##1}}}
\expandafter\def\csname PY@tok@kn\endcsname{\let\PY@bf=\textbf\def\PY@tc##1{\textcolor[rgb]{0.00,0.50,0.00}{##1}}}
\expandafter\def\csname PY@tok@kr\endcsname{\let\PY@bf=\textbf\def\PY@tc##1{\textcolor[rgb]{0.00,0.50,0.00}{##1}}}
\expandafter\def\csname PY@tok@bp\endcsname{\def\PY@tc##1{\textcolor[rgb]{0.00,0.50,0.00}{##1}}}
\expandafter\def\csname PY@tok@fm\endcsname{\def\PY@tc##1{\textcolor[rgb]{0.00,0.00,1.00}{##1}}}
\expandafter\def\csname PY@tok@vc\endcsname{\def\PY@tc##1{\textcolor[rgb]{0.10,0.09,0.49}{##1}}}
\expandafter\def\csname PY@tok@vg\endcsname{\def\PY@tc##1{\textcolor[rgb]{0.10,0.09,0.49}{##1}}}
\expandafter\def\csname PY@tok@vi\endcsname{\def\PY@tc##1{\textcolor[rgb]{0.10,0.09,0.49}{##1}}}
\expandafter\def\csname PY@tok@vm\endcsname{\def\PY@tc##1{\textcolor[rgb]{0.10,0.09,0.49}{##1}}}
\expandafter\def\csname PY@tok@sa\endcsname{\def\PY@tc##1{\textcolor[rgb]{0.73,0.13,0.13}{##1}}}
\expandafter\def\csname PY@tok@sb\endcsname{\def\PY@tc##1{\textcolor[rgb]{0.73,0.13,0.13}{##1}}}
\expandafter\def\csname PY@tok@sc\endcsname{\def\PY@tc##1{\textcolor[rgb]{0.73,0.13,0.13}{##1}}}
\expandafter\def\csname PY@tok@dl\endcsname{\def\PY@tc##1{\textcolor[rgb]{0.73,0.13,0.13}{##1}}}
\expandafter\def\csname PY@tok@s2\endcsname{\def\PY@tc##1{\textcolor[rgb]{0.73,0.13,0.13}{##1}}}
\expandafter\def\csname PY@tok@sh\endcsname{\def\PY@tc##1{\textcolor[rgb]{0.73,0.13,0.13}{##1}}}
\expandafter\def\csname PY@tok@s1\endcsname{\def\PY@tc##1{\textcolor[rgb]{0.73,0.13,0.13}{##1}}}
\expandafter\def\csname PY@tok@mb\endcsname{\def\PY@tc##1{\textcolor[rgb]{0.40,0.40,0.40}{##1}}}
\expandafter\def\csname PY@tok@mf\endcsname{\def\PY@tc##1{\textcolor[rgb]{0.40,0.40,0.40}{##1}}}
\expandafter\def\csname PY@tok@mh\endcsname{\def\PY@tc##1{\textcolor[rgb]{0.40,0.40,0.40}{##1}}}
\expandafter\def\csname PY@tok@mi\endcsname{\def\PY@tc##1{\textcolor[rgb]{0.40,0.40,0.40}{##1}}}
\expandafter\def\csname PY@tok@il\endcsname{\def\PY@tc##1{\textcolor[rgb]{0.40,0.40,0.40}{##1}}}
\expandafter\def\csname PY@tok@mo\endcsname{\def\PY@tc##1{\textcolor[rgb]{0.40,0.40,0.40}{##1}}}
\expandafter\def\csname PY@tok@ch\endcsname{\let\PY@it=\textit\def\PY@tc##1{\textcolor[rgb]{0.25,0.50,0.50}{##1}}}
\expandafter\def\csname PY@tok@cm\endcsname{\let\PY@it=\textit\def\PY@tc##1{\textcolor[rgb]{0.25,0.50,0.50}{##1}}}
\expandafter\def\csname PY@tok@cpf\endcsname{\let\PY@it=\textit\def\PY@tc##1{\textcolor[rgb]{0.25,0.50,0.50}{##1}}}
\expandafter\def\csname PY@tok@c1\endcsname{\let\PY@it=\textit\def\PY@tc##1{\textcolor[rgb]{0.25,0.50,0.50}{##1}}}
\expandafter\def\csname PY@tok@cs\endcsname{\let\PY@it=\textit\def\PY@tc##1{\textcolor[rgb]{0.25,0.50,0.50}{##1}}}

\def\PYZbs{\char`\\}
\def\PYZus{\char`\_}
\def\PYZob{\char`\{}
\def\PYZcb{\char`\}}
\def\PYZca{\char`\^}
\def\PYZam{\char`\&}
\def\PYZlt{\char`\<}
\def\PYZgt{\char`\>}
\def\PYZsh{\char`\#}
\def\PYZpc{\char`\%}
\def\PYZdl{\char`\$}
\def\PYZhy{\char`\-}
\def\PYZsq{\char`\'}
\def\PYZdq{\char`\"}
\def\PYZti{\char`\~}
% for compatibility with earlier versions
\def\PYZat{@}
\def\PYZlb{[}
\def\PYZrb{]}
\makeatother


    % Exact colors from NB
    \definecolor{incolor}{rgb}{0.0, 0.0, 0.5}
    \definecolor{outcolor}{rgb}{0.545, 0.0, 0.0}



    
    % Prevent overflowing lines due to hard-to-break entities
    \sloppy 
    % Setup hyperref package
    \hypersetup{
      breaklinks=true,  % so long urls are correctly broken across lines
      colorlinks=true,
      urlcolor=urlcolor,
      linkcolor=linkcolor,
      citecolor=citecolor,
      }
    % Slightly bigger margins than the latex defaults
    
    \geometry{verbose,tmargin=1in,bmargin=1in,lmargin=1in,rmargin=1in}
    
    

    \begin{document}
    
    
    \maketitle
    
    

    
    \section{PROBLEM 4}\label{problem-4}

    We have \(n\) chains with \(N\) spins in each, obeying periodic boundary
conditions. Hamiltonian:

\begin{equation}
    H=-J_{\parallel}\sum_{m=1}^{n}\sum_{i=1}^{N}\sigma_{m,i}\sigma_{m,i+1} - J_{\perp}\sum_{m=1}^{n}\sum_{i=1}^{N}\sigma_{m,i}\sigma_{m+1,i} -B\sum_{m=1}^{n}\sum_{i=1}^{N}\sigma_{m,i}
\end{equation}

Here \(\sigma_{m, N+1}=\sigma_{m, 1}\) and
\(\sigma_{n+1, i}=\sigma_{1, i}\). This corresponds to wrapping the two
dimensional surface of the chains around a torus.

    \subsubsection{a) Every possible spin-configuration for one
chain}\label{a-every-possible-spin-configuration-for-one-chain}

For a given \(i\) in \(\{1, 2, ..., N\}\), we want to generate every
possible spin state for the \(i\)-th component of the \(n\) chains. One
systematic way to do this, is to consider all binary numbers up to
\(2^{n}\) (these can be expressed with \(n\) digits), and make spin
up/down correspond to \(1\) or \(0\).

    \begin{Verbatim}[commandchars=\\\{\}]
{\color{incolor}In [{\color{incolor}1}]:} \PY{k+kn}{import} \PY{n+nn}{numpy} \PY{k}{as} \PY{n+nn}{np}
        
        \PY{k}{def} \PY{n+nf}{genAllStates}\PY{p}{(}\PY{n}{n}\PY{p}{)}\PY{p}{:}
            \PY{c+c1}{\PYZsh{}int n = number of chains}
            \PY{c+c1}{\PYZsh{}returns: 2\PYZca{}n x n array (int)}
            
            \PY{c+c1}{\PYZsh{}Initially, place all spins up}
            \PY{n}{states} \PY{o}{=} \PY{n}{np}\PY{o}{.}\PY{n}{ones}\PY{p}{(}\PY{p}{(}\PY{l+m+mi}{2}\PY{o}{*}\PY{o}{*}\PY{n}{n}\PY{p}{,} \PY{n}{n}\PY{p}{)}\PY{p}{,} \PY{n}{dtype}\PY{o}{=}\PY{n+nb}{int}\PY{p}{)}
            \PY{n}{intToBin} \PY{o}{=} \PY{k}{lambda} \PY{n}{n}\PY{p}{:} \PY{n+nb}{bin}\PY{p}{(}\PY{n}{n}\PY{p}{)}\PY{p}{[}\PY{l+m+mi}{2}\PY{p}{:}\PY{p}{]}
            \PY{k}{for} \PY{n}{i} \PY{o+ow}{in} \PY{n+nb}{range}\PY{p}{(}\PY{l+m+mi}{2}\PY{o}{*}\PY{o}{*}\PY{n}{n}\PY{p}{)}\PY{p}{:}
                \PY{c+c1}{\PYZsh{}Convert i into a binary string with appropriate amount of zeros in front}
                \PY{n}{b} \PY{o}{=} \PY{n}{intToBin}\PY{p}{(}\PY{n}{i}\PY{p}{)}
                \PY{n}{b} \PY{o}{=} \PY{n}{b}\PY{o}{.}\PY{n}{zfill}\PY{p}{(}\PY{n}{n}\PY{p}{)}
                \PY{k}{for} \PY{n}{j} \PY{o+ow}{in} \PY{n+nb}{range}\PY{p}{(}\PY{n}{n}\PY{p}{)}\PY{p}{:}
                    \PY{c+c1}{\PYZsh{}Subtract the j\PYZhy{}th digit of the binary representation of i for all j spins of spin state no. i}
                    \PY{n}{states}\PY{p}{[}\PY{n}{i}\PY{p}{]}\PY{p}{[}\PY{n}{j}\PY{p}{]} \PY{o}{\PYZhy{}}\PY{o}{=} \PY{l+m+mi}{2}\PY{o}{*}\PY{n+nb}{int}\PY{p}{(}\PY{n}{b}\PY{p}{[}\PY{n}{j}\PY{p}{]}\PY{p}{)}
            \PY{k}{return} \PY{n}{states}
\end{Verbatim}


    In particular, the case n=5 yields:

    \begin{Verbatim}[commandchars=\\\{\}]
{\color{incolor}In [{\color{incolor}2}]:} \PY{n+nb}{print}\PY{p}{(}\PY{n}{genAllStates}\PY{p}{(}\PY{l+m+mi}{5}\PY{p}{)}\PY{p}{)}
\end{Verbatim}


    \begin{Verbatim}[commandchars=\\\{\}]
[[ 1  1  1  1  1]
 [ 1  1  1  1 -1]
 [ 1  1  1 -1  1]
 [ 1  1  1 -1 -1]
 [ 1  1 -1  1  1]
 [ 1  1 -1  1 -1]
 [ 1  1 -1 -1  1]
 [ 1  1 -1 -1 -1]
 [ 1 -1  1  1  1]
 [ 1 -1  1  1 -1]
 [ 1 -1  1 -1  1]
 [ 1 -1  1 -1 -1]
 [ 1 -1 -1  1  1]
 [ 1 -1 -1  1 -1]
 [ 1 -1 -1 -1  1]
 [ 1 -1 -1 -1 -1]
 [-1  1  1  1  1]
 [-1  1  1  1 -1]
 [-1  1  1 -1  1]
 [-1  1  1 -1 -1]
 [-1  1 -1  1  1]
 [-1  1 -1  1 -1]
 [-1  1 -1 -1  1]
 [-1  1 -1 -1 -1]
 [-1 -1  1  1  1]
 [-1 -1  1  1 -1]
 [-1 -1  1 -1  1]
 [-1 -1  1 -1 -1]
 [-1 -1 -1  1  1]
 [-1 -1 -1  1 -1]
 [-1 -1 -1 -1  1]
 [-1 -1 -1 -1 -1]]

    \end{Verbatim}

    \subsubsection{\texorpdfstring{b) Transfer matrix
\(P\)}{b) Transfer matrix P}}\label{b-transfer-matrix-p}

For a single lattice site \(\tau_{i}\):

\begin{equation}
    P_{\tau_{i}\tau_{i+1}} = exp\left[\beta J_{\parallel} \sum_{k=1}^{n} \sigma_{i}^{(k)}\sigma_{i+1}^{(k)} + \beta J_{\perp} \sum_{k=1}^{n}\sigma_{i}^{(k)}\sigma_{i}^{(k+1)} + \frac{\beta B}{2}\sum_{k=1}^{n}(\sigma_{i}^{(k)}+\sigma_{i+1}^{(k)})\right]
\end{equation}

can be expressed as a \(2^{n}\times2^{n}\) -matrix with all the possible
(binary) combinations of \(\sigma_{i}^{(k)}\) and \(\sigma_{i+1}^{(k)}\)
for \(k=\{1, 2, ..., n\}\). I.e.
\(P_{\tau_{i}\tau_{i+1}} \rightarrow P_{l,m}\) for
\(l, m \in \mathbb{N}\).

Using periodic boundary conditions, we sidestep possible issues with the
above sums running out of of indices. Also, there is no inert connection
between lattice sites \(i\) and \(i+1\) before we determine it with the
Hamiltonian, so we may assume the energy contribution between such sites
is the same for all pairs. As such, \(P = P_{\tau_{i}\tau_{j}}\) for
some lattice sites \(\tau_{i}\) and \(\tau_{j}\), and is non-zero for
all \(j=i+1\). Otherwise, \(P\) is independent of \(i\).

    \begin{Verbatim}[commandchars=\\\{\}]
{\color{incolor}In [{\color{incolor}17}]:} \PY{c+c1}{\PYZsh{}Parameters}
         \PY{n}{Jpar} \PY{o}{=} \PY{l+m+mi}{1}
         \PY{n}{Jperp} \PY{o}{=} \PY{l+m+mi}{1}
         \PY{n}{B} \PY{o}{=} \PY{l+m+mf}{1e\PYZhy{}2}
         \PY{n}{beta} \PY{o}{=} \PY{l+m+mf}{0.5} \PY{c+c1}{\PYZsh{}Dummy value}
         
         \PY{c+c1}{\PYZsh{}Matrix element (l,m)}
         \PY{k}{def} \PY{n+nf}{P\PYZus{}lm}\PY{p}{(}\PY{n}{l}\PY{p}{,} \PY{n}{m}\PY{p}{,} \PY{n}{n}\PY{p}{,} \PY{n}{beta}\PY{p}{)}\PY{p}{:}
             \PY{c+c1}{\PYZsh{}int l, m = row \PYZam{} column position in P}
             \PY{c+c1}{\PYZsh{}int n = number of chains}
             \PY{c+c1}{\PYZsh{}returns: float}
             
             \PY{c+c1}{\PYZsh{}Select a possible spin state given by l and k (NB! 0\PYZhy{}indexed)}
             \PY{n}{states} \PY{o}{=} \PY{n}{genAllStates}\PY{p}{(}\PY{n}{n}\PY{p}{)}
             \PY{n}{sig} \PY{o}{=} \PY{k}{lambda} \PY{n}{l}\PY{p}{,} \PY{n}{k}\PY{p}{:} \PY{n}{states}\PY{p}{[}\PY{n}{l}\PY{p}{]}\PY{p}{[}\PY{n}{k}\PY{p}{]}
             \PY{c+c1}{\PYZsh{}Sums in Hamiltonian}
             \PY{n}{Spar}\PY{p}{,} \PY{n}{Sperp}\PY{p}{,} \PY{n}{Smag} \PY{o}{=} \PY{l+m+mi}{0}\PY{p}{,} \PY{l+m+mi}{0}\PY{p}{,} \PY{l+m+mi}{0}
             \PY{k}{for} \PY{n}{k} \PY{o+ow}{in} \PY{n+nb}{range}\PY{p}{(}\PY{n}{n}\PY{p}{)}\PY{p}{:}
                 \PY{n}{Spar} \PY{o}{+}\PY{o}{=} \PY{n}{sig}\PY{p}{(}\PY{n}{l}\PY{p}{,} \PY{n}{k}\PY{p}{)}\PY{o}{*}\PY{n}{sig}\PY{p}{(}\PY{n}{m}\PY{p}{,} \PY{n}{k}\PY{p}{)}
                 \PY{n}{Sperp} \PY{o}{+}\PY{o}{=} \PY{n}{sig}\PY{p}{(}\PY{n}{l}\PY{p}{,} \PY{n}{k}\PY{p}{)}\PY{o}{*}\PY{n}{sig}\PY{p}{(}\PY{n}{m}\PY{p}{,} \PY{p}{(}\PY{n}{k}\PY{o}{+}\PY{l+m+mi}{1}\PY{p}{)}\PY{o}{\PYZpc{}}\PY{k}{n})
                 \PY{n}{Smag} \PY{o}{+}\PY{o}{=} \PY{n}{sig}\PY{p}{(}\PY{n}{l}\PY{p}{,} \PY{n}{k}\PY{p}{)} \PY{o}{+} \PY{n}{sig}\PY{p}{(}\PY{n}{m}\PY{p}{,} \PY{n}{k}\PY{p}{)}
             \PY{k}{return} \PY{n}{np}\PY{o}{.}\PY{n}{exp}\PY{p}{(} \PY{n}{beta}\PY{o}{*}\PY{n}{Jpar}\PY{o}{*}\PY{n}{Spar} \PY{o}{+} \PY{n}{beta}\PY{o}{*}\PY{n}{Jperp}\PY{o}{*}\PY{n}{Sperp} \PY{o}{+} \PY{l+m+mi}{1}\PY{o}{/}\PY{l+m+mi}{2}\PY{o}{*}\PY{n}{beta}\PY{o}{*}\PY{n}{B}\PY{o}{*}\PY{n}{Smag} \PY{p}{)}
         
         \PY{c+c1}{\PYZsh{}Generate matrix}
         \PY{k}{def} \PY{n+nf}{P}\PY{p}{(}\PY{n}{n}\PY{p}{,} \PY{n}{beta}\PY{p}{)}\PY{p}{:}
             \PY{c+c1}{\PYZsh{}int n = number of chains}
             \PY{c+c1}{\PYZsh{}returns: 2\PYZca{}n x 2\PYZca{}n array (float)}
             
             \PY{n}{mat} \PY{o}{=} \PY{n}{np}\PY{o}{.}\PY{n}{zeros}\PY{p}{(}\PY{p}{(}\PY{l+m+mi}{2}\PY{o}{*}\PY{o}{*}\PY{n}{n}\PY{p}{,} \PY{l+m+mi}{2}\PY{o}{*}\PY{o}{*}\PY{n}{n}\PY{p}{)}\PY{p}{)}
             \PY{k}{for} \PY{n}{l} \PY{o+ow}{in} \PY{n+nb}{range}\PY{p}{(}\PY{l+m+mi}{2}\PY{o}{*}\PY{o}{*}\PY{n}{n}\PY{p}{)}\PY{p}{:}
                 \PY{k}{for} \PY{n}{m} \PY{o+ow}{in} \PY{n+nb}{range}\PY{p}{(}\PY{l+m+mi}{2}\PY{o}{*}\PY{o}{*}\PY{n}{n}\PY{p}{)}\PY{p}{:}
                     \PY{n}{mat}\PY{p}{[}\PY{n}{l}\PY{p}{]}\PY{p}{[}\PY{n}{m}\PY{p}{]} \PY{o}{=} \PY{n}{P\PYZus{}lm}\PY{p}{(}\PY{n}{l}\PY{p}{,} \PY{n}{m}\PY{p}{,} \PY{n}{n}\PY{p}{,} \PY{n}{beta}\PY{p}{)}
             \PY{k}{return} \PY{n}{mat}
\end{Verbatim}


    NB: The matrices \(P\) are symmetric, so one could save time by only
computing the upper/lower triangular part of \(P\).

    \subsubsection{c) Eigenvalues}\label{c-eigenvalues}

Now, I plot (the real part of) all the eigenvalues of a single \(P\) for
\(n\) chains (there are \(n\) of them) as a function of \(\beta\). The
parameters are all set to unity.

    \begin{Verbatim}[commandchars=\\\{\}]
{\color{incolor}In [{\color{incolor}23}]:} \PY{k+kn}{import} \PY{n+nn}{matplotlib}
         \PY{n}{matplotlib}\PY{o}{.}\PY{n}{rcParams}\PY{p}{[}\PY{l+s+s1}{\PYZsq{}}\PY{l+s+s1}{figure.figsize}\PY{l+s+s1}{\PYZsq{}}\PY{p}{]} \PY{o}{=} \PY{p}{[}\PY{l+m+mi}{12}\PY{p}{,} \PY{l+m+mi}{7}\PY{p}{]}
         \PY{n}{matplotlib}\PY{o}{.}\PY{n}{rcParams}\PY{o}{.}\PY{n}{update}\PY{p}{(}\PY{p}{\PYZob{}}\PY{l+s+s1}{\PYZsq{}}\PY{l+s+s1}{font.size}\PY{l+s+s1}{\PYZsq{}}\PY{p}{:} \PY{l+m+mi}{22}\PY{p}{\PYZcb{}}\PY{p}{)}
         \PY{k+kn}{import} \PY{n+nn}{matplotlib}\PY{n+nn}{.}\PY{n+nn}{pyplot} \PY{k}{as} \PY{n+nn}{plt}
         
         \PY{c+c1}{\PYZsh{}Plot in beta}
         \PY{k}{def} \PY{n+nf}{plotOverBeta}\PY{p}{(}\PY{n}{n}\PY{p}{)}\PY{p}{:}
             \PY{c+c1}{\PYZsh{}int n = number of chains}
             \PY{c+c1}{\PYZsh{}returns: Nan}
             
             \PY{n}{num} \PY{o}{=} \PY{l+m+mi}{100}
             \PY{n}{betaRange} \PY{o}{=} \PY{n}{np}\PY{o}{.}\PY{n}{linspace}\PY{p}{(}\PY{l+m+mf}{0.005}\PY{p}{,} \PY{l+m+mf}{1.0}\PY{p}{,} \PY{n}{num}\PY{p}{)}
             \PY{n}{plotVals} \PY{o}{=} \PY{n}{np}\PY{o}{.}\PY{n}{zeros}\PY{p}{(}\PY{p}{(}\PY{n}{num}\PY{p}{,}\PY{l+m+mi}{2}\PY{o}{*}\PY{o}{*}\PY{n}{n}\PY{p}{)}\PY{p}{)}
             \PY{c+c1}{\PYZsh{}Solve eigenvalues for every value of beta in the given range}
             \PY{k}{for} \PY{n}{i} \PY{o+ow}{in} \PY{n+nb}{range}\PY{p}{(}\PY{n}{num}\PY{p}{)}\PY{p}{:}
                 \PY{n}{s}\PY{p}{,} \PY{n}{S} \PY{o}{=} \PY{n}{np}\PY{o}{.}\PY{n}{linalg}\PY{o}{.}\PY{n}{eigh}\PY{p}{(}\PY{n}{P}\PY{p}{(}\PY{n}{n}\PY{p}{,} \PY{n}{betaRange}\PY{p}{[}\PY{n}{i}\PY{p}{]}\PY{p}{)}\PY{p}{)}
                 \PY{n}{plotVals}\PY{p}{[}\PY{n}{i}\PY{p}{]} \PY{o}{=} \PY{n}{np}\PY{o}{.}\PY{n}{real}\PY{p}{(}\PY{n}{s}\PY{p}{)}
             \PY{c+c1}{\PYZsh{} We end up with 2**n different eigenvalue\PYZhy{}plots}
             \PY{k}{for} \PY{n}{i} \PY{o+ow}{in} \PY{n+nb}{range}\PY{p}{(}\PY{l+m+mi}{2}\PY{o}{*}\PY{o}{*}\PY{n}{n}\PY{p}{)}\PY{p}{:}
                 \PY{n}{plt}\PY{o}{.}\PY{n}{semilogy}\PY{p}{(}\PY{n}{betaRange}\PY{p}{,} \PY{n}{plotVals}\PY{p}{[}\PY{p}{:}\PY{p}{,}\PY{n}{i}\PY{p}{]}\PY{p}{,} \PY{n}{lw}\PY{o}{=}\PY{l+m+mi}{2}\PY{p}{)}
             \PY{n}{plt}\PY{o}{.}\PY{n}{grid}\PY{p}{(}\PY{p}{)}
             \PY{n}{plt}\PY{o}{.}\PY{n}{xlabel}\PY{p}{(}\PY{l+s+sa}{r}\PY{l+s+s1}{\PYZsq{}}\PY{l+s+s1}{\PYZdl{}}\PY{l+s+s1}{\PYZbs{}}\PY{l+s+s1}{beta\PYZdl{}}\PY{l+s+s1}{\PYZsq{}}\PY{p}{)}
             \PY{n}{plt}\PY{o}{.}\PY{n}{ylabel}\PY{p}{(}\PY{l+s+sa}{r}\PY{l+s+s1}{\PYZsq{}}\PY{l+s+s1}{Eigenval}\PY{l+s+s1}{\PYZsq{}}\PY{p}{)}
             \PY{n}{plt}\PY{o}{.}\PY{n}{show}\PY{p}{(}\PY{p}{)}
             \PY{k}{return} \PY{k+kc}{None}
             
         \PY{n}{plotOverBeta}\PY{p}{(}\PY{l+m+mi}{5}\PY{p}{)}
\end{Verbatim}


    \begin{Verbatim}[commandchars=\\\{\}]
/home/svein/anaconda3/lib/python3.6/site-packages/matplotlib/ticker.py:2198: UserWarning: Data has no positive values, and therefore cannot be log-scaled.
  "Data has no positive values, and therefore cannot be "

    \end{Verbatim}

    \begin{center}
    \adjustimage{max size={0.9\linewidth}{0.9\paperheight}}{output_10_1.png}
    \end{center}
    { \hspace*{\fill} \\}
    
    We notice that the largest eigenvalue is the largest for all values of
\(\beta\). This motivates our choice to approximate \(Z\) as being only
the largest eigenvalue.

    \subsubsection{d)}\label{d}

Using the relations given in the exercise text, one can express \(m\)
like:

\begin{equation}
    m = \frac{1}{n}\frac{\partial}{\partial (\beta B)} ln(\lambda_{n, \text{max}}) = \frac{1}{nB}\frac{\partial}{\partial \beta} ln(\lambda_{n, \text{max}}).
\end{equation}

    \begin{Verbatim}[commandchars=\\\{\}]
{\color{incolor}In [{\color{incolor}24}]:} \PY{k}{def} \PY{n+nf}{generateLargestLambda}\PY{p}{(}\PY{n}{n}\PY{p}{)}\PY{p}{:}
             \PY{n}{num} \PY{o}{=} \PY{l+m+mi}{100}
             \PY{n}{betaRange} \PY{o}{=} \PY{n}{np}\PY{o}{.}\PY{n}{linspace}\PY{p}{(}\PY{l+m+mf}{0.005}\PY{p}{,} \PY{l+m+mf}{1.0}\PY{p}{,} \PY{n}{num}\PY{p}{)}
             \PY{n}{Z} \PY{o}{=} \PY{n}{np}\PY{o}{.}\PY{n}{zeros}\PY{p}{(}\PY{p}{(}\PY{n}{num}\PY{p}{,}\PY{l+m+mi}{2}\PY{o}{*}\PY{o}{*}\PY{n}{n}\PY{p}{)}\PY{p}{)}
             \PY{k}{for} \PY{n}{i} \PY{o+ow}{in} \PY{n+nb}{range}\PY{p}{(}\PY{n}{num}\PY{p}{)}\PY{p}{:}
                 \PY{n}{s}\PY{p}{,} \PY{n}{S} \PY{o}{=} \PY{n}{np}\PY{o}{.}\PY{n}{linalg}\PY{o}{.}\PY{n}{eigh}\PY{p}{(}\PY{n}{P}\PY{p}{(}\PY{n}{n}\PY{p}{,} \PY{n}{betaRange}\PY{p}{[}\PY{n}{i}\PY{p}{]}\PY{p}{)}\PY{p}{)}
                 \PY{c+c1}{\PYZsh{}Add eigenvalues to Z}
                 \PY{n}{Z}\PY{p}{[}\PY{n}{i}\PY{p}{]} \PY{o}{=} \PY{n}{np}\PY{o}{.}\PY{n}{real}\PY{p}{(}\PY{n}{s}\PY{p}{)}
             \PY{c+c1}{\PYZsh{}Return the entire column containing the largest eigenvalue}
             \PY{k}{return} \PY{n}{Z}\PY{p}{[}\PY{p}{:}\PY{p}{,}\PY{n}{np}\PY{o}{.}\PY{n}{argmax}\PY{p}{(}\PY{n}{Z}\PY{p}{)}\PY{o}{\PYZpc{}}\PY{k}{2}**n]
         
         \PY{k}{def} \PY{n+nf}{m}\PY{p}{(}\PY{n}{n}\PY{p}{)}\PY{p}{:}
             \PY{n}{Z} \PY{o}{=} \PY{n}{generateLargestLambda}\PY{p}{(}\PY{n}{n}\PY{p}{)}
             \PY{k}{return} \PY{l+m+mi}{1}\PY{o}{/}\PY{p}{(}\PY{n}{n}\PY{o}{*}\PY{n}{B}\PY{p}{)}\PY{o}{*}\PY{n}{np}\PY{o}{.}\PY{n}{gradient}\PY{p}{(}\PY{n}{np}\PY{o}{.}\PY{n}{log}\PY{p}{(}\PY{n}{Z}\PY{p}{)}\PY{p}{)}
         
         \PY{k}{def} \PY{n+nf}{plotm}\PY{p}{(}\PY{p}{)}\PY{p}{:}
             \PY{k}{for} \PY{n}{n} \PY{o+ow}{in} \PY{n+nb}{range}\PY{p}{(}\PY{l+m+mi}{2}\PY{p}{,}\PY{l+m+mi}{6}\PY{p}{)}\PY{p}{:}
                 \PY{n}{mag} \PY{o}{=} \PY{n}{m}\PY{p}{(}\PY{n}{n}\PY{p}{)}
                 \PY{n}{betaRange} \PY{o}{=} \PY{n}{np}\PY{o}{.}\PY{n}{linspace}\PY{p}{(}\PY{l+m+mf}{0.005}\PY{p}{,} \PY{l+m+mf}{1.0}\PY{p}{,} \PY{l+m+mi}{100}\PY{p}{)}
                 \PY{n}{plt}\PY{o}{.}\PY{n}{plot}\PY{p}{(}\PY{n}{betaRange}\PY{p}{,} \PY{n}{mag}\PY{p}{,} \PY{n}{label}\PY{o}{=}\PY{n}{n}\PY{p}{,} \PY{n}{lw}\PY{o}{=}\PY{l+m+mi}{2}\PY{p}{)}
             \PY{n}{plt}\PY{o}{.}\PY{n}{xlabel}\PY{p}{(}\PY{l+s+sa}{r}\PY{l+s+s1}{\PYZsq{}}\PY{l+s+s1}{\PYZdl{}}\PY{l+s+s1}{\PYZbs{}}\PY{l+s+s1}{beta\PYZdl{}}\PY{l+s+s1}{\PYZsq{}}\PY{p}{)}
             \PY{n}{plt}\PY{o}{.}\PY{n}{ylabel}\PY{p}{(}\PY{l+s+sa}{r}\PY{l+s+s1}{\PYZsq{}}\PY{l+s+s1}{m}\PY{l+s+s1}{\PYZsq{}}\PY{p}{)}
             \PY{n}{plt}\PY{o}{.}\PY{n}{grid}\PY{p}{(}\PY{p}{)}
             \PY{n}{plt}\PY{o}{.}\PY{n}{legend}\PY{p}{(}\PY{p}{)}
             \PY{n}{plt}\PY{o}{.}\PY{n}{show}\PY{p}{(}\PY{p}{)}
         
         \PY{n}{plotm}\PY{p}{(}\PY{p}{)}
\end{Verbatim}


    \begin{center}
    \adjustimage{max size={0.9\linewidth}{0.9\paperheight}}{output_13_0.png}
    \end{center}
    { \hspace*{\fill} \\}
    
    \subsubsection{e)}\label{e}

Using the given relations, I calculate and plot

\begin{equation}
\frac{C_{B}(T)}{Nnk_{B}} = \frac{\beta^{2}}{n}\frac{\partial^{2} ln(\lambda_{n, \text{max}})}{\partial \beta^{2}}
\end{equation}

over \(\beta\).

    \begin{Verbatim}[commandchars=\\\{\}]
{\color{incolor}In [{\color{incolor}25}]:} \PY{k}{def} \PY{n+nf}{plotC\PYZus{}B}\PY{p}{(}\PY{p}{)}\PY{p}{:}
             \PY{k}{for} \PY{n}{n} \PY{o+ow}{in} \PY{n+nb}{range}\PY{p}{(}\PY{l+m+mi}{2}\PY{p}{,} \PY{l+m+mi}{6}\PY{p}{)}\PY{p}{:}
                 \PY{n}{betaRange} \PY{o}{=} \PY{n}{np}\PY{o}{.}\PY{n}{linspace}\PY{p}{(}\PY{l+m+mf}{0.005}\PY{p}{,} \PY{l+m+mf}{1.0}\PY{p}{,} \PY{l+m+mi}{100}\PY{p}{)}
                 \PY{n}{Z} \PY{o}{=} \PY{n}{generateLargestLambda}\PY{p}{(}\PY{n}{n}\PY{p}{)}
                 \PY{n}{plotvals} \PY{o}{=} \PY{l+m+mi}{1}\PY{o}{/}\PY{n}{n} \PY{o}{*} \PY{n}{betaRange}\PY{o}{*}\PY{o}{*}\PY{l+m+mi}{2}\PY{o}{*}\PY{n}{np}\PY{o}{.}\PY{n}{gradient}\PY{p}{(}\PY{n}{np}\PY{o}{.}\PY{n}{gradient}\PY{p}{(}\PY{n}{np}\PY{o}{.}\PY{n}{log}\PY{p}{(}\PY{n}{Z}\PY{p}{)}\PY{p}{)}\PY{p}{)}
                 \PY{n}{plt}\PY{o}{.}\PY{n}{plot}\PY{p}{(}\PY{n}{betaRange}\PY{p}{,} \PY{n}{plotvals}\PY{p}{,} \PY{n}{label}\PY{o}{=}\PY{n}{n}\PY{p}{,} \PY{n}{lw}\PY{o}{=}\PY{l+m+mi}{2}\PY{p}{)}
             \PY{n}{plt}\PY{o}{.}\PY{n}{xlabel}\PY{p}{(}\PY{l+s+sa}{r}\PY{l+s+s1}{\PYZsq{}}\PY{l+s+s1}{\PYZdl{}}\PY{l+s+s1}{\PYZbs{}}\PY{l+s+s1}{beta\PYZdl{}}\PY{l+s+s1}{\PYZsq{}}\PY{p}{)}
             \PY{n}{plt}\PY{o}{.}\PY{n}{ylabel}\PY{p}{(}\PY{l+s+sa}{r}\PY{l+s+s1}{\PYZsq{}}\PY{l+s+s1}{\PYZdl{}}\PY{l+s+s1}{\PYZbs{}}\PY{l+s+s1}{frac}\PY{l+s+s1}{\PYZob{}}\PY{l+s+s1}{C\PYZus{}}\PY{l+s+si}{\PYZob{}B\PYZcb{}}\PY{l+s+s1}{(T)\PYZcb{}}\PY{l+s+s1}{\PYZob{}}\PY{l+s+s1}{Nnk\PYZus{}}\PY{l+s+si}{\PYZob{}B\PYZcb{}}\PY{l+s+s1}{\PYZcb{}\PYZdl{}}\PY{l+s+s1}{\PYZsq{}}\PY{p}{)}
             \PY{n}{plt}\PY{o}{.}\PY{n}{grid}\PY{p}{(}\PY{p}{)}
             \PY{n}{plt}\PY{o}{.}\PY{n}{legend}\PY{p}{(}\PY{p}{)}
             \PY{n}{plt}\PY{o}{.}\PY{n}{show}\PY{p}{(}\PY{p}{)}
         
         \PY{n}{plotC\PYZus{}B}\PY{p}{(}\PY{p}{)}
\end{Verbatim}


    \begin{center}
    \adjustimage{max size={0.9\linewidth}{0.9\paperheight}}{output_15_0.png}
    \end{center}
    { \hspace*{\fill} \\}
    
    The dependence of \(C\) on \(B\):

Large \(|B| \implies C\rightarrow 0\) as the enthalpy is completely
dominated by the term \(B\sum_{m=1}^{n}\sum_{i=1}^{N}\sigma_{m,i}\) in
the Hamiltonian. Any increase in \(T\) contributes to kinetic energy,
but completely dies out in comparison.

Small \(|B|\) means all the terms in \(H\) are fairly small, so
increasing kinetic energy can give a noticable contribution to enthalpy.

All in all; \(C(B)\) decreases symmetrically around the origin.


    % Add a bibliography block to the postdoc
    
    
    
    \end{document}
