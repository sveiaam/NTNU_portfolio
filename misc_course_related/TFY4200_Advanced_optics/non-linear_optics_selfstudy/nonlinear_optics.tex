\documentclass{article}
%Characters
\usepackage[utf8]{inputenc}
%Figures
\usepackage{graphicx}
%Links and references
\usepackage{hyperref}
%Math
\usepackage{bm}
\usepackage{amsmath}
%Physics
\usepackage{physics}
%Referencing automatically detects type of reference + multi-figure referencing
\usepackage{cleveref}


\title{TFY4200 - Problem set 7 (6) \\
A self study on non-linear optics}
\author{Svein Åmdal}
\date{20. March 2020}

\begin{document}
\maketitle

\section*{Non-linearity of scalar fields}
Previously, we have taken the dielectric response to an applied $E$-field (generally from a EM-wave) to be linear, meaning that $D=\epsilon_{0}\epsilon E$ holds. Now, we focus on the dielectric polarization $P$ and assume it might be non-linear, i.e.

\begin{equation}
\label{Polarization_expansion}
P(t) = \epsilon_{0} \sum_{n=1}^{\infty} \chi^{(n)} E^{n}(t) = \sum_{n=1}^{\infty}P^{(n)}(t).
\end{equation}

$\{\chi^{(n>1)}\}$ is called the \textit{higher order non-linear optical suceptibilites}.

\subsection*{Second-order non-linear optical processe}

For example, if a laser whose field is given by

\begin{equation}
\label{di-chromatic}
E_{\text{inc}}(t) = E_{1}e^{-i\omega_{1}t} + E_{2}e^{-i\omega_{2}t} + E_{1}^{*}e^{i\omega_{1}t} + E_{2}^{*}e^{i\omega_{2}t}
\end{equation}

is incident on a crystal with $\chi^{(2)} \neq 0$, we calculate

\begin{equation}
\begin{aligned}
P^{(2)}(t) &= \epsilon_0\chi^{(2)}E^2 (t) \\
& = \epsilon_{0}\chi^{(2)} \bigg[ E_1^2 e^{-2i\omega_1 t} + E_2^2 e^{-2i\omega_2 t} + E_1^{*2} e^{2i\omega_1 t} + E_2^{*2} e^{2i\omega_2 t} \\
& + 2E_1 E_2 e^{-i(\omega_1 + \omega_2)t} + 2E_1^{*} E_2^{*} e^{i(\omega_1 + \omega_2)t} + 2E_1 E_2^{*} e^{-i(\omega_1 - \omega_2)t} \\
& + 2E_1^{*} E_2 e^{i(\omega_1 - \omega_2)t} + 2E_1 E_1^{*} + 2E_2 E_2^{*} \bigg] \\
& := \sum_n P(\omega_n)e^{-i\omega_n t}.
\end{aligned}
\end{equation}

We now identify

\begin{align}
P(2\omega_1) & = \epsilon_0\chi^{(2)}E_1^2 \label{SHG1} \\
P(2\omega_2) & = \epsilon_0\chi^{(2)}E_2^2, \label{SHG2}
\end{align}

as \textit{second-harmonic generation} (SHG) of polarization response. These contributions to the polarization response has twice the incoming frequency, and they stem from interaction between only a single frequency (either $\omega_1$ or $\omega_2$). Thus, SHG is also apparent if the incoming wave consists of only a single frequency component.

Further, we identify

\begin{equation}
\label{SFG}
P(\omega_1 + \omega_2) = 2\epsilon_0\chi^{(2)}E_1 E_2,
\end{equation}

as \textit{sum-frequency generation} (SFG) of polarization response, and

\begin{equation}
\label{DFG}
P(\omega_1 - \omega_2) = 2\epsilon_0\chi^{(2)}E_1 E_2^* , 
\end{equation}

as \textit{difference-frequency generation} (DFG) the polarization response. Note that \crefrange{SHG1}{DFG} all have corresponding negative frequency components, which we ommit.

Finally, we may identify

\begin{equation}
\label{OR}
P(0) = 2\epsilon_0\chi^{(2)} \bigg[ E_1 E_1^* + E_2 E_2^* \bigg],
\end{equation}

as the DC components of $P$, often called \textit{optical rectification}, a quantity that is also present with only a single incoming frequency.

If the crystal has few avaliable states with the bandgap energies $\hbar \omega_{\{\text{inc}\}}$, almost all the power of the incident wave is immediately radiated with SHG frequency $2\omega_{\text{inc}}$, or with one of the sum- or difference-generated frequencies. One application of this is to produce tunable radiation. If the input frequencies are produced by one fixed-$\omega$ and one tunable-$\omega$ laser, the resulting radiation is tunable, and may be in a different frequency regime than the tunable laser.

\subsection*{Third-order non-linear optical processes}

For simplicity, let the incoming field be monochromatic, given by

\begin{equation}
E_{\text{inc}}(t) = E \cos(\omega t).
\end{equation}

Using a mathematical cosine identity, we find

\begin{equation}
\begin{aligned}
P^{(3)}(t) & = \epsilon_0\chi^{(3)}E_{\text{inc}}^3 (t) \\
& = \frac{1}{4}\epsilon_0\chi^{(3)}E^3 (t) \bigg[ \cos(3\omega t) + 3\cos(\omega t) \bigg].
\end{aligned}
\end{equation}
From \cref{Polarization_expansion}, we have that the third-order polarization response is

\begin{equation}
P^{(3)}(t) = \epsilon_0\chi^{(3)}E^3 (t).
\end{equation}
Similarly to the second-harmonic generation of polarization, we now get a \textit{third-harmonic generation} of polarization, namely the $3\omega$-term.

For the more general case of a tri-chromatic incoming wave, we get generation of all possible frequencies that are sums and/or differences between the different incoming frequencies.

The processes discussed so far are all \textit{parametric}, meaning that they do not alter the quantum state of the crystal (in any meaningful way, at least). If the final state of the crystal has a sustantial number of excited electrons compared to the initial state, the process is \textit{non-parametric}, and must be described by a complex $\chi$. In a non-parametric process, the photon energy may not be conserved, as it may become absorbed in the medium. This is similar to the familiar case of a complex refractive index being a measure of wave dissipation in the medium.


\section*{Non-linearity of vector fields (formal treatment)}

We begin by writing out the electric field as a sum of frequency components (both positive and negative), i.e.

\begin{equation}
\label{E_freq_comp}
\bm{E}(\bm{r},t) = \sum_n \bm{E}_n (\bm{r}, t) =  \sum_n \bm{A}_n e^{i(\bm{k}_n \cdot \bm{r} - \omega_n t)},
% + \bm{A}_n^* e^{-i(\bm{k}_n \cdot \bm{r} - \omega_n t)}
\end{equation}

where

\begin{equation}
\bm{A}_n = \bm{A}(\omega_n) = \bm{A}(-\omega_n) = |\bm{E}_n (\bm{r})| = \frac{1}{2}E.
\end{equation}

(The factor 1/2 comes from counting both positive and negative $\omega$ in the sum.) Now, the second order susceptibility tensor $\chi_{ijk}^{(2)}$ is defined in terms of the cartesian components $i$, $j$ and $k$ of $\bm{E}$ by the equation

\begin{equation}
P_i(q_{mn} = \epsilon_0 \sum_{j}\sum_{k}\sum_{m,n | q_{mn}=\text{const.}} \chi_{ijk}^{(2)}(q_{mn}; \omega_m, \omega_n )E_j (\omega_m )E_k(\omega_n ),
\end{equation}

where $q_{mn}$ is shorthand for $\omega_m + \omega_n$, and it is assumed to be fixed in the sum over $m$ and $n$.

By letting $m,n \in \{1,2\}$ and $q = \omega_1 + \omega_2$, one automatically recieves sum-frequency generation of a response with frequency $\omega_1 + \omega_2$ exactly like before.

By letting $\omega_1$ be given, and letting $q = 2\omega_1$, one automatically recieves second-harmonic generation of response polarization, also like before.

Precisely analoguosly, the third order susceptibility tensor $\chi_{ijkl}^{(3)}$ is defined by

\begin{equation}
P_i(q_{mno}) = \epsilon_0 \sum_{j,k,l}\sum_{m,n,o | q_{mno}=\text{const.}} \chi_{ijkl}^{(3)}(q_{mno}; \omega_m , \omega_m , \omega_o )E_j (\omega_m )E_k(\omega_n )E_l(\omega_o),
\end{equation}

where $q_{mno} = \omega_m + \omega_n + \omega_o$. In both cases we may perform the summation over $m,n$ (and $o$), and collect the result in a multiplicative \textit{degeneracy factor}, $D$.

\section*{Anharmonic oscillator}
The classical harmonic oscillator for a position-like parameter $x$ only has terms up to second order in $x$. The first order term may be removed by a coordinate transformation, and the constant term may be removed by carefully selecting the zero-level of the harmonic potential. Let's now introduce a $x^3$-anharmonic term. Thus the oscillator potential becomes
\begin{equation}
U(x) = \frac{1}{2}m\omega_0^2 x^2 + \frac{1}{3}ma x^3,
\end{equation}
for some choices of parameters $m$, $\omega_0$ and $a$.
The anharmonic force from displacement in this potential becomes of second order in $x$, as
\begin{equation}
F = -\frac{\partial}{\partial x}U(x) = -m\omega_0^2 x - max^2.
\end{equation}
Inserting this "external" force into the classical equation of motion for an electron (charge $e$) with a damping factor $\gamma$, and an external, di-chromatic $E$-field given by \cref{di-chromatic}, we solve the modified equation of motion,

\begin{equation}
\ddot{x} + 2\gamma\dot{x}+\omega_0^2x + ax^2 = -\lambda\frac{e E(t)}{m},
\end{equation}

in terms of a perturbation expansion for small $x$ in the now introduced expansion parameter $\lambda$, i.e.
\begin{equation}
x = \sum_n \lambda^n x^{(n)}.
\end{equation}

Collecting the terms of equal order in $\lambda$, we find that the $1^{\text{st}}$ order terms correspond to the unperturbed equation of motion. Thus

\begin{equation}
\label{x1_ugly}
\begin{aligned}
x^{(1)}(t) = & -\frac{e}{m}\frac{E_1}{\omega_0^2-\omega_1^2-2\gamma i\omega_1}\exp{-i\omega_1 t} \\
& -\frac{e}{m}\frac{E_2}{\omega_0^2-\omega_2^2-2\gamma i\omega_2}\exp{-i\omega_2 t} \\
&-\frac{e}{m}\frac{E_1}{\omega_0^2-\omega_1^2+2\gamma i\omega_1}\exp{i\omega_1 t} \\
& -\frac{e}{m}\frac{E_2}{\omega_0^2-\omega_2^2+2\gamma i\omega_2}\exp{i\omega_2 t}.
\end{aligned}
\end{equation}

The $2^{\text{nd}}$ order terms collect to form the equation

\begin{equation}
\label{x2_def}
\ddot{x}^{(2)} + 2\gamma\dot{x}^{(2)} + \omega_0^2x^{(2)} + a(x^{(1)})^2 = 0.
\end{equation}

Inserting \cref{x1_ugly} into \cref{x2_def}, we obtain an expression for $x^{(2)}$. If one defines

\begin{equation}
\label{D_def}
D(\omega_j) := \omega_0^2 - \omega_j^2 - 2\gamma i\omega_j,
\end{equation}

one may write down the frequency components of $x^{(2)}$ instead.

Given the known linear polarization

\begin{equation}
\label{lin_pol_def}
P^{(1)}(\omega) = \epsilon_0\chi^{(1)}(\omega)E(\omega) = -Nex^{(1)}(\omega),
\end{equation}

we may calculate

\begin{equation}
\label{chi_1}
\chi^{(1)}(\omega_j) = \frac{Ne^2/m}{\epsilon_0 D(\omega_j)}
\end{equation}

based on \crefrange{x1_ugly}{lin_pol_def}. Defining non-linear polarization as

\begin{equation}
P^{(2)}(q_{12}) = 2\epsilon_0\chi^{(2)}(q_{12}; \omega_1, \omega_2)E(\omega_1)E(\omega_2) = -Nex^{(2)}(q_{12}),
\end{equation}

still using the shorthand $q_{ij} = \omega_i + \omega_j$,  we analogously find

\begin{align}
\chi^{(2)}(q_{12}; \omega_1, \omega_2) & = \frac{Nae^3/m^2}{\epsilon_0 D(q_{12})D(\omega_1)D(\omega_2)} \\
& = \frac{\epsilon_0^2ma}{N^2e^3}\chi^{(1)}(q_{12})\chi^{(1)}(\omega_1)\chi^{(1)}(\omega_2),
\end{align}

where we in the last equality have used \cref{chi_1}. This corresponds to sum-frequency generation. A special case of this is when $\omega_1=\omega_2=\omega$. Then

\begin{equation}
\chi^{(2)}(2\omega;\omega,\omega) = \frac{\epsilon_0^2ma}{N^2e^3}\chi^{(1)}(2\omega)[\chi^{(1)}(\omega)]^2,
\end{equation}

and the polarization resulting from this is akin to second-harmonic generation.

We may also have $\omega_2 \rightarrow -\omega_2$. Then,
\begin{equation}
\chi^{(2)}(q_{1,-2};\omega_1, -\omega_2) = \frac{\epsilon_0^2ma}{N^2e^3}\chi^{(1)}(q_{1,-2})\chi^{(1)}(\omega_1)\chi^{(1)}(-\omega_2),
\end{equation}

and we have arrived at difference-frequency generation of polarization.

A third special case is when $\omega_2 = -\omega_1 := -\omega$. Then

\begin{equation}
\chi^{(2)}(0,\omega,-\omega) = \frac{\epsilon_0^2ma}{N^2e^3}\chi^{(1)}(0)\chi^{(1)}(\omega)\chi^{(1)}(-\omega)
\end{equation}

represents optical rectification.

A similar analysis is performed for the tensor form of $\chi^{(2)}$ to obtain

\begin{equation}
\begin{aligned}
\chi_{ijkl}^{(2)}(q_{mno}, \omega_m, \omega_n, \omega_o) = & \frac{bm\epsilon_0^3}{3N^3e^4}\chi^{(1)}(q_{mno})\chi^{(1)}(\omega_m)\chi^{(1)}(\omega_n)\chi^{(1)}(\omega_o) \\
& \times \bigg[ \delta_{ij}\delta_{kl}+\delta_{ik}\delta_{jl}+\delta_{il}\delta_{jk} \bigg],
\end{aligned}
\end{equation}

where $\chi_{ijkl}$ means the susceptibility in direction $i$ from $\bm{E}$ with components in directions $j$, $k$ and $l$.

\end{document}